\edef\figdir{\currfiledir/fig}

\chapter{Rethomics} \label{rethomics}

\epigraph{`I suppose it is tempting, if the only tool you have is a hammer, to treat everything as if it were a nail.'}{--- Abraham Maslow,\\
	\emph{The Psychology of Science}~\cite[p.~15]{maslow_psychology_1966}}
\section{Background}


Tools such as the ethoscope platform, that I described in the previous chapter, and others allow biologists to gather a large amount of behavioural data from multiple animals.
Some authors have suggested the use of the term `ethomics', an analogy between the high-throughput acquisition and analysis of behavioural time series and other large-scale phenotyping approaches such as transcriptomics, proteomics and metabolomics\cite{reiser_ethomics_2009}.
The transition to ethomics era is extremely promising as it paves the way for in-depth quantitative analyses which, in turn, leads to the characterisation of new principles and ultimately a better understanding of biology\cite{brown_study_2017}.

Most methods that are suffixed with the term `omics'\emd{}sometimes, I concede, generously so\footnote{see Jonathan Eisen's `Worst New Omics Word Award: Museomics' post\\  \url{https://phylogenomics.blogspot.ca/2009/01/worst-new-omics-word-award-museomics.html}}\emd{}share, at least, two features that, I would argue, are equally crucial. 
Firstly, they rely on technological breakthroughs that have, over the last few decades, improved throughput and reduced cost of data acquisition by orders of magnitude. 
Secondly, they adopt advances in data sciences, data management and software engineering to provide communities with sets of maintained analysis tools and databases which intercommunicate and share standards (\eg{} file formats, programming languages and software libraries).

In the last decade, several commercial and academic platforms were developed to acquire  various behaviours such as activity\cite{faville_how_2015}, position\cite{pelkowski_novel_2011} and feeding\cite{itskov_automated_2014,ro_flic_2014} of single or multiple\cite{swierczek_high-throughput_2011,perez-escudero_idtracker_2014} animals over long durations (days or weeks).
However, regarding the subsequent analysis of the results, there is less effort in the development of a high-level, unified, programmatic framework that could be used, for instance, to create pipelines.

For instance, in the field of sleep and circadian rhythm, analysis tools are generally graphical user interfaces built in a `top-down' manner.
That is, they are designed to achieve a set of predefined functionalities in the scope anticipated by their developers.
In other words, instead of focussing on defining general data structures and concepts to then specify them to their own cases, authors have generally opted for a more rigid approach \cite{gilestro_pysolo_2009,schmid_new_2011,cichewicz_shinyr-dam_2018}.

In contrast, in the  bioinformatics community, many widely adopted tools are instead built in a `bottom-up' fashion and opt for either a command-line or a programming interface.
At the foundation level, \glspl{api} are designed and provide a well-documented set of tools for developers to build libraries.
Then, these libraries are used by individual researchers to write scripts, or can even serve as a back-end to build user interfaces.
This approach favours collaboration in favour of replicated work and ultimately promotes maintenance of software (\eg{} if and when the original authors retire from the field). 
Programming interfaces are generally more robust.
In addition, they are flexible since they can be used in conjunction with a set of other tools to build a pipeline.
For this reason, state-of-the-art analysis and visualisation tools are based on programming interfaces\cite{wickham_ggplot2_2016}.

The type of data studied in traditional bioinformatics is discrete both in space and state (\ie{} nucleotide and amino acids) and conveniently ubiquitous, which certainly encourages a shared toolbox. 
In ethomics, the data gathered could be, at first, assumed to be prohibitively inconsistent to justify sharing tools (\eg{} what do fish movements and fruit fly feeding behaviours share?)
However, I argue that behavioural data is conceptually largely agnostic of the acquisition platform and paradigm. 	
Typically, the behaviour of each experimental individual is described by a long time series (possibly multivariate and irregular).
Importantly, individuals are labelled with arbitrary metadata defined by the experimenter (\emph{e.g.} sex, treatment and genotype).
At a low level, linking combining and processing metadata and data from hundreds of individuals, each recorded for days or weeks, is not a trivial challenge, and the community may benefit from shared solutions.

In the chapter herein, I present \texttt{rethomics}, an efficient and flexible tool in \texttt{R} to unify the analysis of behavioural data.
Firstly, I will describe a general design and workflow.
Secondly, I will detail my computational solution to store and manipulate behavioural data.
Thirdly, I will conceptualise a general principle to import ethomics results.
Fourthly, I will explain how I extended the concept of the grammar of graphics\cite{wilkinson_grammar_2005,wickham_ggplot2_2016} to the visualisation of behavioural data.
Finally, I will briefly present some of the functionalities of \texttt{rethomics} to the analysis of sleep and circadian rhythm.

Waiting for peer review, I have made available the manuscript describing \texttt{rethomics} as a preprint\cite{geissmann_rethomics_2018} (see availability section~\ref{sec:rethomics-availability}). 


\section{Workflow}

\texttt{rethomics} is implemented as a collection of complementary \texttt{R} packages (fig.~\ref{fig:workflow}).
This architecture follows the model of modern frameworks such as the \texttt{tidyverse}\cite{wickham_tidyverse_2017}, which results in enhanced maintainability and testability,
as each step of the analysis workflow (\emph{i.e.} data import, manipulation and visualisation) is handled by a different package.

\begin{figure}[h!]
  \centering   
   \includegraphics[width=0.95\textwidth]{\currfiledir/.\currfilebase.pdf}
  \caption[The rethomics workflow]{\ctit{The rethomics workflow.}
  		Diagram representing the interplay between, from left to right, the raw data, the \texttt{rethomics} packages (in blue) and the rest of the \texttt{R} ecosystem.
		This figure is adapted from my own work\cite{geissmann_rethomics_2018}.
  	\label{fig:\currfilebase}
  }
\end{figure}







The \texttt{behavr} package lies at the centre of \texttt{rethomics}.
It was designed as a flexible and efficient solution to store large amounts of data (\emph{e.g.} position and activity) as well as metadata (\emph{e.g.} treatment and genotype) in a single \texttt{data.table}-derived object\cite{dowle_data.table_2017}.
Input packages all import experimental data as a \texttt{behavr} table which can, in turn, be analysed and visualised regardless of the original input platform.
Results and plots integrate seamlessly into the \texttt{R} ecosystem, thus providing users with state-of-the-art visualisation and statistical tools.


\section{Internal data structure}
\texttt{behavr} (fig.~\ref{fig:behavr}), is a new data structure, based on the widely adopted \texttt{data.table} object\cite{dowle_data.table_2017}.
It is designed to address two challenges that are inherent to handling ethomics results.
Firstly, behavioural time series can be very long as one or multiple variables may be sampled\emd{}sometimes heterogeneously\emd{}several times a second, for days or weeks.
Each behavioural series could represent variables that encode the activity, position, orientation, size, colour and so on. 
Secondly, a large number of individuals are is studied (often several hundreds of animals).
In order to statistically account for covariates\emd{}for instance, in the context of full factorial experiments\emd{}it is important that each individual is associated with `metadata'.
The metadata is a set of `metavariables' that describe all experimental conditions.

\begin{figure}[h!]
	\centering   
	\includegraphics[width=0.95\textwidth]{\currfiledir/.\currfilebase.pdf}
	  \caption[The behavr data structure]{\ctit{The behavr data structure.}
	Schematical representation of a \texttt{behavr} object, the central data structure in \texttt{rethomics}.
	In the metadata (left), each row refers to one experimental individual. 
	Columns are metavariables that can be either \emph{required}(\ie{} defined by the acquisition platform) or \emph{user-defined} (\emph{i.e.} arbitrary).
	In the data (right), each row represents a time point, that is the information about one individual at one time-point.
	Data and metadata are internally joined on the \texttt{id} field, the unique identifier of an experimental individual.
	The column names used in this figure are only examples which will differ in practice. 
	This figure is adapted from my own work\cite{geissmann_rethomics_2018}.
%	B: Non exhaustive list of uses of a \texttt{behavr} table (referred as \texttt{dt}). 
%	In addition to operations on data, which are inherited from \texttt{data.table},
%	we provide utilities designed specifically to act on both metadata and data.  
%	Commented examples are prefixed by \texttt{>}.
	\label{fig:\currfilebase}
}
\end{figure}


A \texttt{behavr} object contains a pair of tables: metadata and data.
The two tables are semantically linked by the extension of the \texttt{data.table} syntax, thus rendering the manipulation, the joining and the accession of metadata transparent.
In other words, this architecture allows users to map data to its parent metadata without explicit knowledge of databases, in a statistical environment (fig.~\ref{fig:behavr2}).
For instance, when  data is filtered, only the remaining individuals are left in the metadata. 
It is also important that metadata and data can interoperate.
For example, when one wants to update a variable according to the value of a metavariable (say, alter the variable $x$ only for animals with the metavariable $sex=`male'$).

\begin{figure}[h!]
	\centering   
	\includegraphics[width=0.95\textwidth]{\currfiledir/.\currfilebase.pdf}
	  \caption[Summary of operations in \texttt{behavr}]{\ctit{Summary of operations in \texttt{behavr}.}
	Summary of the functionalities provided by \texttt{behavr} data structure.
	Code examples (prefixed by \texttt{>}) are provided and commented.
	All operations supported by \texttt{data.table} are inherited.
	Additional utilities are specifically implemented to manipulate both metadata and data together.  
	This figure is adapted from my own work\cite{geissmann_rethomics_2018}.
	\label{fig:\currfilebase}
}
\end{figure}



\section{Linking and loading data}

Metadata encapsulates the information describing exhaustively and unambiguously each experimental individual.
As such, it must be provided by the user and constitute a precise record of the experiment.
I reasoned that is the if the experimenter was required to provide a metadata table describing each individual,
it should also include information that can be parsed to allocate a unique identifier and fetch the data in the file system.
Indeed, the process of finding the data for each animal, when multiple devices are used and when experiments are repeated several times, is prohibitively error-prone and hardly reproducible.
Although result files are organised in a very different manner across acquisition platform, conceptually
the operation of enriching the metadata\emd{}which I call `\emph{linking}'\emd{}with technical details to, later, fetch the matching data\emd{}which I name `\emph{loading}'\emd{}is general and can be abstracted (fig.~\ref{fig:linking}).

\begin{figure}[h!]
	\centering   
	\includegraphics[width=0.95\textwidth]{\currfiledir/.\currfilebase.pdf}
	  \caption[Linking and loading]{\ctit{Linking and loading.}
\textbf{A}, Linking. The mandatory fields in the user-provided metadata are used to automatically locate the corresponding files.
Location data is stored as an additional field. This process also checks for existence and  integrity of the data.
Individuals are also automatically allocated a unique identifier during this process.
\textbf{B}, Loading. 
The linked metadata is then used to fetch and optionally pre-process data.
All data is grouped and eventually stored, with the metadata in a single \texttt{behavr} object.
\label{fig:\currfilebase}
}
\end{figure}


Naturally, each acquisition platform will have a set of mandatory metavariables that are necessary for linking, which are clearly stated the documentation.
Currently, I provide a package to read single or multibeam \gls{dam} data and another one for ethoscope files\cite{geissmann_ethoscopes_2017}.

This linking and loading approach has several advantages.
Firstly, the metadata file is a standardised structure which allows different users to collaborate using a shared format.
Secondly, since each individual is a row in the metadata, users can\emd{}and are encouraged\emd{}to intersperse experimental conditions (otherwise, experimenters tend to confound metavariables\emd{}for example, one genotype per machine and multiple machines).
Furthermore, this framework facilitates the addition of new replicates insofar as additional individuals will just be represented by new rows\emd{}and the date of the replicate can be statistically accounted as a covariate.

\section{Visualisation}

Visualisation is essential for behavioural researchers who use it to control the quality of their data,
summarise population trends and generate hypotheses.
In \texttt{R}, the widely adopted \texttt{ggplot2}\cite{wickham_ggplot2_2016} package implements and extends the concept of the grammar of graphics\cite{wilkinson_grammar_2005}, providing users with a very high-level syntax for data visualisation.

Since behavioural time series can be prohibitively large, preprocessing is often needed before visualisation.
In addition, \texttt{ggplot2} does not have access to metadata, which are often needed since they encapsulate biological questions.
In order to interface the \texttt{rethomics} framework with \texttt{ggplot2} I developed \texttt{ggetho},
which is fully compatible with \texttt{ggplot}, but also provides new layers and scales to represent behaviours.
Figure~\ref{fig:ggetho-example} illustrates the use of \texttt{ggetho} in a restricted use case\emd{}and more examples are available on the \texttt{rethomics} website.

\begin{figure}[h!]
	\centering   
	\includegraphics[width=0.95\textwidth]{\currfiledir/.\currfilebase.pdf}
	  \caption[Visualisation with \texttt{ggetho}]{\ctit{Visualisation with \texttt{ggetho}.}
  	Code snippet (left) and resulting plot (right) for an example visualisation of population trend. 
  	The arrows indicate where each code expression is represented on the plot.
  	In this example, the input data, \texttt{dt}, is a behavr table, with variables \texttt{t} (continuous) and \texttt{moving} (logical),
	and metavariables \texttt{treatment} (`Control' or `Treated') and \texttt{sex} (`M' or `F').
    The \texttt{ggetho()} function, the \texttt{stat\_pop\_etho} and \texttt{stat\_ld\_annotations} layers, and the \textbf{scale\_x\_days} scale are part of \texttt{ggetho}, but integrate seamlessly 
    with \texttt{ggplot}, for instance with the faceting mechanism shown here (\ie{} \texttt{facte\_grid}).
	\label{fig:\currfilebase}
}
\end{figure}


\section{Sleep and circadian analysis}

\texttt{rethomics} is not exclusively designed for the analysis of sleep and circadian rhythm
but, in the context of my research\emd{}and in order to attract users and collaborators\emd{}, I endeavoured to reproduce and extend some of the visualisation tools and computational methods used by my community.
In the manuscript describing \texttt{rethomics}, I provide a reproducible code example to illustrate how it can be used to perform `canonical' circadian analysis\cite{geissmann_rethomics_2018}.
In this short section, I will succinctly show some of the main functionalities, and refer the reader to the documentation for a comprehensive description.
Specifically, I will show examples of `double-plotted actograms' and periodograms generated in the \texttt{rethomics} framework.

Double-plotted actograms are widely used by chronobiologists to visually assess the regularity and period of individual behaviours.
\texttt{ggetho} provides a generalised implementation to `multi-plot' time series (\ie{} double-plotted, triple-plotted and so on).
Importantly, the faceting functionality of \texttt{ggplot} makes is possible to sort and draw immediately one plot per individual (fig.~\ref{fig:circadian-analysis}A).

\begin{figure}[h!]
	\centering   
	\includegraphics[width=0.95\textwidth]{\currfiledir/.\currfilebase.pdf}
	  \caption[Analysis of the periodicity in the activity in DAMs]{\ctit{Analysis of the periodicity in the activity in DAMs.}
	  	Example of visualisation and analysis of the circadian periodicity in \droso{} activity with \texttt{rethomics}.
	  	Eight selected animals are shown on different facets and labelled alphabetically from a to h.
\textbf{A}, Double-plotted actograms. 
	The height of the bars represent the relative average activity binned over time (here, 30~min bins, the default).
	The Y axis shows the time onset, in days, and the X, the time after the onset, in hours. 
	Each value is repeated on the next row to facilitate visualisation (\ie{} avoid edge effects).
\textbf{B}, $\chi^2$ periodograms for the activity data visualised in \textbf{A}.
	The blue `$\times$' symbols represent the most significant peak (if present).
	The red line shows significance at $\alpha = 0.05$ (the default).
	Note that a right slant in \textbf{A} corresponds a peak period greater than 24~h in \textbf{B}.
	This figure is modified from my own work\cite{geissmann_rethomics_2018}.
	\label{fig:\currfilebase}
}
\end{figure}


In addition to the double-plotted actograms, which are mostly for quality control and illustrative purposes, 
circadian analysis often requires the computation of periodograms,
the most widely used being the $\chi^2$\cite{sokolove_chi_1978} and the Lomb-Scargle\cite{ruf_lomb-scargle_1999}  methods.
In the \texttt{zeitgebr} package, I implemented several periodogram methods.
In addition, I endeavoured to formalise their output so that further analysis can be directly compared and reiterated with different functions.
I also provided tools to automatically find significant peaks as well as  display and summarise periodograms.
For instance, figure~\ref{fig:circadian-analysis}B shows the $\chi^2$ periodograms, and their respective peaks, matching the plots in figure~\ref{fig:circadian-analysis}A.

The \texttt{sleepr} package, in \texttt{rethomics}, provides several routines for the analysis of sleep behaviour in \droso{}.
In particular, it can be used to apply the five-minute rule (even in heterogeneously sample data),
it implements the ethoscope definition of movement and velocity correction presented in the previous chapter and
has a method to encode categorical behavioural variables as `bouts' (\ie{} onset and duration).

\section{Availability and future directions}
\label{sec:rethomics-availability}
I implemented \texttt{rethomics} in \texttt{R}\cite{r_core_team_r_2017} since it is widely taught and adopted by statisticians and computational biologists. 
The packages developed during my thesis are hosted on CRAN, the official repository.
An overall exhaustive tutorial is publicly available at \url{http://rethomics.github.io}.
This whole documentation is automatically compiled and deployed upon upstream software update,
which ensures reproducibility of the examples.
According to state-of-the-art practices, all packages are also individually documented,
continuously integrated and unit tested on several versions of \texttt{R}.

\newpage
\section{Summary}

\begin{itemize}
	\item \texttt{rethomics} provides a set of packages that integrate with one another and with  \texttt{R}.
	\item  The \texttt{behavr} object is an efficient and flexible data structure that stores both variables and metavariables.
	\item The `linking and loading' approach is a consistent and exhaustive method to associate experimental conditions to behavioural data.
	\item The \texttt{ggetho} package extends \texttt{ggplot} to visualise ethomics data
	\item \texttt{rethomics} already implements a working set of tools for circadian and sleep quantification, and could develop to include other aspects of behavioural analysis.
\end{itemize}



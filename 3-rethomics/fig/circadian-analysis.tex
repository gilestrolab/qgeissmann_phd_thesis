\begin{figure}[h!]
	\centering   
	\includegraphics[width=0.95\textwidth]{\currfiledir/.\currfilebase.pdf}
	  \caption[Analysis of the periodicity in the activity in DAMs]{\ctit{Analysis of the periodicity in the activity in DAMs.}
	  	Example of visualisation and analysis of the circadian periodicity in \droso{} activity with \texttt{rethomics}.
	  	Eight selected animals are shown on different facets and labelled alphabetically from a to h.
\textbf{A}, Double-plotted actograms. 
	The height of the bars represent the relative average activity binned over time (here, 30~min bins, the default).
	The Y axis shows the time onset, in days, and the X, the time after the onset, in hours. 
	Each value is repeated on the next row to facilitate visualisation (\ie{} avoid edge effects).
\textbf{B}, $\chi^2$ periodograms for the activity data visualised in \textbf{A}.
	The blue `$\times$' symbols represent the most significant peak (if present).
	The red line shows significance at $\alpha = 0.05$ (the default).
	Note that a right slant in \textbf{A} corresponds a peak period greater than 24~h in \textbf{B}.
	This figure is modified from my own work\cite{geissmann_rethomics_2018}.
	\label{fig:\currfilebase}
}
\end{figure}

\begin{figure}[h!]
	\centering   
	\includegraphics[width=0.95\textwidth]{\currfiledir/.\currfilebase.pdf}
	  \caption[Daytime sleep deprivation causes rebound when it finished before dusk]{\ctit{Daytime sleep deprivation causes rebound when it finished before dusk.}
	\textbf{A}, Proportion of time engaged in either of the three behavioural states (q: quiescence, m: micro-movement and w:walking) and relative position (from the food, 0, to the cotton wool, 1) 
	are represented as different rows.
	Females and males are shown on the left and right, respectively.
	The grey rectangle in the background,
	\textbf{between ZT=0 and ZT=8~h}, %%%%%%%%%
	coincides with the permissive time window of stimulus delivery,
	Stimuli were delivered to animals each time they had been immobile for 20 consecutive seconds. %%%%%%%%
	Controls animals (grey) were undisturbed and hosted in neighbouring tubes, interspersed with the treated individuals (plum).
	\textbf{B}, Average number of stimulus delivered in each consecutive 30~min.
	\textbf{C}, Extra quiescence during the 3~h of rebound (blue rectangle in the background of \textbf{A}), expressed in extra minutes compared to the predicted quiescence (see method subsection~\ref{subsec:mm-rebound}).
	The shaded areas around the average lines, in \textbf{A} and \textbf{B}, and the error bars in \textbf{C} are 95\% bootstrap resampling confidence intervals on the mean.
	$N_{sex,interval} > 60~\forall~sex \times treatment$.
\label{fig:\currfilebase}
}
\end{figure}
